\documentclass[a4paper,12pt]{article}
%\documentclass[a4paper,10pt]{scrartcl}

\usepackage{amsmath,amssymb}
\usepackage{epic}
\usepackage{eepic}
\usepackage{bm}
\usepackage{array}
\usepackage{float}
\usepackage{multirow}
\usepackage{fancyhdr}

\usepackage{vmargin}            % red�finir les marges
\setmarginsrb{2cm}{0.5cm}{2cm}{1cm}{0cm}{1cm}{0cm}{1cm}


\title{Potential Flow Around a Regular Body in 2D}
\author{}
\date{}

\begin{document}
\maketitle
\section{Settling the Problem}

Suppose the fluids flows by given vector velocity field $u$. The velocity field can be written as the gradient 
$\nabla \phi$ of a velocity potential $\phi$. 
\begin{equation}\label{Pr0}
 u=\nabla \phi
\end{equation}

Let we assume that vector field $u$ to be irrotational. It means 
$curl(u)=0$. We can also write
$\nabla\times u=0$.

We consider the flow to be incompressible, we have $\nabla\cdot u=0$. If we substitute ~\eqref{Pr0}, we have
\begin{align*}
&\nabla \cdot u=0 \\
 &\nabla \cdot \nabla\phi=0 \\
 &\Delta \phi=0
\end{align*}
From here we get:
\begin{equation} \label{Pr1}
 \Delta \phi=0
\end{equation}

Let consider fluid in an infinite domain. We have the velocity $u_0$ along $x$ in one direction.  This conditions can be written as
\begin{equation} \label{Pr2}
u=u_0\vec{e_x}
\end{equation}
From ~\eqref{Pr1} and ~\eqref{Pr2}, we have problem:


\begin{equation} \label{problem1}
\begin{split}
\Delta \phi &=0\\
u=\nabla \phi &\to u_0 \vec{e_x} \quad \text{as } x\to\pm\infty
\end{split}
\end{equation}
This problem has obvious solution i.e $\phi= u_0 x$.

Let we add a regular obstacle $\Omega_0$. We consider $\partial\Omega_0$ to be $C^1$. It means the obstacle doesn't have angle nor discontinuity. 
On the border of the obstacle, the velocity is tangential.
\begin{equation}\label{eq1}
u\cdot n=0
\end{equation}
with $n$ is the (exterior) normal vector of $\Omega_0$.
We substitute $u=\nabla\Phi$ to ~\eqref{eq1}, we get 
\begin{equation}\label{Pr3}
\nabla\phi\cdot n=0
\end{equation}
Equation ~\eqref{Pr3} is called Homogeneous Neumann Boundary Condition.  
From equation ~\eqref{Pr0},~\eqref{Pr1}, ~\eqref{Pr2} and ~\eqref{Pr3}, we obtain the problem:
\begin{equation} \label{problem2}
\begin{split}
\Delta \phi=0 \quad \text{On: } \mathbb{R}^2 \setminus \overline{\Omega_0}\\
u=\nabla\phi\to u_0\vec{e_x}\quad \text{as } x \to \pm\infty\\
\nabla\phi \cdot n=0 \quad \partial \Omega_0=\Gamma_0
\end{split}
\end{equation}
We have to find the harmonic function that satisfies all the boundary conditions.

Without the obstacle, problem ~\eqref{problem1} has solution $u_0x$. We can use it to find $\phi$. Let we do the changing variable in order to get the solution of problem ~\eqref{problem2}.
Let $\psi=\phi-u_0 x$ and $\phi$ is the actual solution. 

\begin{align*}
\psi&=\phi-u_0 x\\
 \Delta \psi &= \Delta \phi-\Delta (u_0x)\\
\Delta \psi &= \Delta \phi\\
\nabla \psi &= \nabla\phi-u_0\vec{e_x}
\end{align*}
when $x\to\pm\infty$ value of $\nabla \psi\to 0$.
From here, we get 
\begin{equation}
 \nabla\psi\to0 \quad \text{as } x\to\pm\infty
\end{equation}
From the last condition in ~\eqref{problem2} we get
\begin{align*}
 \nabla\psi\cdot n &= \nabla(\phi-u_0 x)\cdot n \text{on } \Gamma_0\\
&=\nabla\phi \cdot n - \nabla(u_0 x)\cdot n\\
&=0-u_0\nabla x\cdot n\\
&=-u_0 \vec{e_x} \cdot n\\
&=-u_0 n_x
\end{align*}

The problem after changing variable become
\begin{equation}  \label{problem3}
\begin{split}
\Delta \psi&=0 \quad \text{in } \mathbb{R}^2 \setminus \overline{\Omega_0}\\
\nabla\psi &\to 0 \quad \text{as } x\to\pm\infty\\
\nabla\psi \cdot n &= -u_0 n_x \quad \text{on } \Gamma_0 
\end{split}
\end{equation}



\end{document}
